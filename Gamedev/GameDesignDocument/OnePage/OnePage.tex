\documentclass[10pt,a4paper,oneside]{article}
\usepackage[portuguese]{babel}
\usepackage[utf8]{inputenc}
\usepackage[T1]{fontenc}
\usepackage{graphicx}
\graphicspath{{images/}}
\usepackage{indentfirst}
\usepackage{hyperref}
\usepackage{color}
\usepackage{vhistory}
\usepackage{fancyhdr}

\pagestyle{fancy}

\definecolor{blue}{RGB}{41,5,195}
\setlength{\parindent}{1.3cm}
\setlength{\parskip}{0.2cm}


\hypersetup{
    %pagebackref=true,
        pdftitle={Template de One Page}, 
        pdfauthor={Lucas Samuel Vieira},
        pdfsubject={Template de página-única do projeto "Template", da Companhia},
        pdfkeywords={Game Design}{Companhia}{Projeto}{Game}{Indie Game}{Jogos}{Jogos Digitais},
        pdfproducer={Lucas Samuel Vieira},
        pdfcreator={Lucas Samuel Vieira},
        colorlinks=true,
        linkcolor=blue,
        citecolor=blue,
        filecolor=magenta,
        urlcolor=blue,
        bookmarksdepth=4
}

% Cabeçalho
\fancyhf{}
\renewcommand{\headrulewidth}{0pt}
\fancyhead[L]{\small PROJETO}
\fancyhead[R]{\small Página-única}

% Rodapé
\renewcommand{\footrulewidth}{1pt}
\fancyfoot[L]{\small ©2016 Companhia}
\fancyfoot[C]{\small \thepage}
\fancyfoot[R]{\small \today}


% Início do Documento
\begin{document}

% Introdução ao OnePage
{\center \huge Página-Única de Projeto de Jogo}

% Histórico de versões
\begin{versionhistory}
    \vhEntry{1.0}{31/08/2016}{LSV}{Criado}
\end{versionhistory}

\section{Dados gerais do jogo}
\begin{itemize}
    \item \textbf{Título:} Nome do Jogo
    \item \textbf{Plataformas Pretendidas:} Xbox 360, Playstation 3
    \item \textbf{Faixa Etária:} De 0 a 99 anos
    \item \textbf{Classificação:} Livre; Everyone
    \item \textbf{Produtos Concorrentes:} Jogo 1, Jogo 2
\end{itemize}

\section{Resumo da História}
O pintinho queria soltar pum, não tinha traseiro e explodiu.

\section{Modos de Gameplay}

\section{Diferenciais de Venda}
\begin{itemize}
    \item Cerca de cinco bullet points;
    \item Como se fosse para a caixa do jogo;
    \item "Funcionalidade X"\ surpreendente/tocante/qualquercoisa não contam.
    \item Mais um item;
    \item Mais um item.
\end{itemize}

\section{Disposições Extras}
Outros pontos que não entram no escopo inicial, mas tangem ao quesito de ideias
adicionais futuras, entram aqui.

\end{document}
